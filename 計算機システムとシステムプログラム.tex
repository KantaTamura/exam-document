\chapter{計算機システムとシステムプログラム}
計算機システム分野:
数の表現,演算制御,命令実行制御,記憶制御,入出力制御
システムプログラム分野:
プロセス管理,処理装置管理,記憶管理,入出力管理,ファイル管理

\section{オペレーティングシステム}

\subsection{OSの役割}

\begin{enumerate}
    \item ハードウェア機構の隠蔽
    \item ハードウェア装置の管理
    \item ソフトウェア実行時の保護
\end{enumerate}

\section{プロセス管理}

ここで,プロセスとはプログラムとメモリに割り付けられるプロセス領域のことである.
イメージとしては\fref{fig:process}のような,プロセス領域の箱に格納されている任意のプログラムで,プロセス領域が異なれば,プログラムが同じでも別のプロセスとして扱われる.

\fig{0.4}{process.drawio.png}{プロセスのイメージ}{process}

プロセス管理とは,OSによって行われる次のような処理である.

\begin{itemize}
    \item プロセス状態の管理
    \item 実行しているプロセスの切り替え(プロセススイッチ) 
    \item プロセスの生成と消去
    \item プロセス実行順序の決定(プロセススケジューリング)
    \item プロセスからの実行フローの生成と管理
    \item 複数プロセスの同期
    \item 複数プロセス間の通信
\end{itemize}

\subsection{プロセス状態}

\fig{0.4}{process_state.drawio.png}{プロセス状態の遷移}{process_state}

\subsection{プロセススケジューリング}

\subsection{プロセス同期}

\subsection{プロセス間通信}

\section{メモリ管理}

\section{入出力制御}
